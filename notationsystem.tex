%part
% 2024-02-16.tex

\chapter{NOTATION SYSTEM}

The customary method of musical notation, which is a product of the \say{trial and error} method, is inadequate for the analysis and study of rhythmic patterns. It offers no common basis for computations. The history of creative experience in music shows that even the greatest composers have been unnecessarily limited in their rhythmic patterns because they thought in terms of ordinary musical notation.\footnote{If, from experience outside the musical field, you already know how graphs are used, it will be sufficient to say at this point that (a) music can be graphed by allowing the lengths of a number of horizontal lines to stand for the \textit{durations} of tones, and causing the distance up or down (ordinate) to stand for the \textit{pitch} levels of the tones; and (b) when graphing duration only, as in these studies, the end of one duration and beginning of the next may be indicated by a \say{turn} (phase change) in the line, as shown in Figure 47. (Ed.)}

The arrangement of tiem-durations, which constitutes the theory of rhythm, may be studied through three parallel systems of notation: (1) numbers, (2) graphs, (3) musical notes.

Understanding the nature of these group formations helps us to compose, to arrange any given musical material, and to play the most involved rhythmic patterns.

Number values will be used in this system in their normal mathematical operations (such as the four actions---addition, subtraction, multiplication, and division---, raising to powers, extracting roots, permutations, etc.)\footnote{Although Schillinger makes much use of mathematics in this work, the reader is not presumed to be a student of mathematics. Each mathematical operation is carefully explained so that those who possess the most elementary knowledge of mathematics will not encounter difficulty either in understanding the text or in performing the necessary operations. (Ed.)}

\section{Graphing Music}

The graph method used in this system is the general method of graphs, \textit{i.e.}, a record of variation of special components, such as pitch or intensity in music, stocks in finance, diseases in medicine, etc. during a given time-period.
