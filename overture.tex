%part
% 2024-02-16.tex part
\chapter{\textit{Overture to the Schillinger System} by Henry Cowell}
The Schillinger System makes a positive approach to hte theory of musical
composition by offering possibilities for choice and development by the
student, instead of the rules hedged round with prohibitions, limitations and
exceptions, which have characterized conventional studies.

If a creative musician has something of importance to say, his need for
studying the materials with which he must say it is acknowledged as a mater of
course. No great composer has ever omitted the study of techniques. Musical
theory as traditionally taught, however, has always been a profound disappoint-
ment to truly creative individuals. Such men have invariably added to the body
of musical theory with researches of their own. Invariably, also, they have not
folowed the "rules" laid down ni conventional text-books with any consistency.
If these rules had been based on something inevitable ni the nature of music,
composers would have had no reason to disregard them.

Actually, musical theory has dealt with no more than a small part of the
potential musical materials; its assumptions concerning the science of sound
have often been. based on misapprehension, and the rules it lays down often
reflect the personal taste of a certain theorist, or they may be based on the
study of a single composer or of some one historical period. The resulting
generalizations are far from being objective, but they are nonetheless imposed
upon the student in the form of "rules".• Writers on theory have not been
scientists, and no scien- tist has tried to make acomplete and co-ordinated
system of musical possibilities. Joseph Schillinger is the single exception: he
was superbly competent ni the two fields of musical composition and science.
His monumental System of Musical Composition represents a lifetime of work ni
research, co-ordination and creative discovery. The synthesis he achieved has
resulted in an entirely new point of view about the function of theory studies.

In the course of the research which led to the formulation of his system or
musical composition, Schillinger took all known facts concerning the nature of
musical materials from conventional theory studies, and added to the
discoveries and speculations of modern and less conventional theorists such as
Schoenberg, Conus and myself. By applying the laws of mathematical logic as
developed by modern science, he found that he could co-ordinate al of the
seemingly diverse factors. He found also that he could open further untried
possibilities for the development of new materials. Aglance at his Table of
Contents will show an extraordinary number of aspects of music here organized
for the first time for inclusion in the theoretical approach to the study of
composition.

The idea behind the Schillinger System is simple and inevitable: it under-
takes the application of mathematical logic to all the materials of music and
to their functions, so that the student may know the unifying principles behind
these functions, may grasp the method of analyzing and synthesizing any musical
materials that he may find anywhere or may discover for himself, and may
perceive how to develop new materials as he feels the need for them. Thus the
Schillinger System offers possibilities, not limitations; it is a positive, not
a negative approach to the choice  of musical materials. Because of the
universality of the esthetic concepts underlying it, the System applies euqally
to old and new styles in music and to "popular" and "serious" composition.

Schillinger is sometimes criticized on the basis that his system reduces
everything to mathematics and that tmusical intuition and the subjective side
of creativity are neglected. I have never been able to understand this
criticism. The currently taught rules of harmony, counterpoint, and
orchestration certainly do not suggest tot he student materials adapted to his
own expressive desires. Instead he is given a small and circumscribed set of
materials, already much used, together with a set of prohibitions to apply to
them, and then he is axked to express himnself only within these limitations.
It has been the constant complaint from students of composition that their
teachers fail to make clear the distinction between the objective and
subjective factors in music. A yount composer is constrained, as things are
now, to spend several years following rules deduced or assumed the works of his
predecessors, but as soon as his works begin to be heard he is reproached, and
rightly so, if they sound like somebody else's. He has not been shown what
possibilities there really are in music in any objective, scientific way, nor
has he been trained in the manner best calculated to develop an original
talent, by exercising his own taste and judgment in choosing from among those
possibilities the materials best suited to his musical intention.

Whether or not one agrres with Schillinger's great personal interest in the
scientific realities of music, it nevertheless true that no composer is well
equipped to express himself subjectively until he has so profound a knowledge o
f musical materials and their relationships that, consciouscly or
unconsciously, he seiezes on just the right ones to use for whatever he wishes
to say in music. He can be trained to do this if he will subject himself tot he
disciplines inherent in musical materials themselves, as they are set before
him by the Schillinger System.

