%part
% 2024-02-16.tex

\chapter{PRELIMINARY REMARKS ON THE THEORY OF RHYTHM}

The Theory of Rhythm is the foundation of Schillinger's system. Bur for him, rhythm is not simply a matter of time-rhythm, which is what is ordinarily meant by the term. Schillinger begins by applying rhythm to time durations, and then exztends it to all other phases of composition---the way in which block-harmonies change, intervals in scales and melody, entrances of counterthemes in counterpoint, distribution of parts through a score, and other processes of composition. Schillinger's statements are clear provided the reader takes the trouble to work them out, rather than merely read them. It must be borne in mind at this stage that the individual processes work out in this book are \textit{all to be used} in the actual composition of music.

\textit{The Schillinger System of Musical Composition} has the integrated construction of a closely reasoned work of science of mathematics. Beginning with Book I, \textit{Theory of Rhythm}, Schillinger successively presents techniques relating to the various phases of comp-osition. Book II develops the \textit{Theory of Pitch Scales}; Book IV, \textit{Melody}; Book V, \textit{Harmony}; Book VI, \textit{Correlation of Melody and Harmony}; Book VII, \textit{Counterpoint}; etc

Mastery o the materials of any one of these books will provide the student with undreamed-of new resources. However, the Schillinger system places its emphasis on \textit{composition}, that is, on the procedure for integrating elements and structures, and not on the detached and uncoordinated techniques. The method for integrating the individual techniques is presented in Book XI, \textit{Theory of Composition}, which is the crowning summit of this work, as the Theory of Rhythm is its foundation.
